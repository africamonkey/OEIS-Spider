% XeLaTeX

\documentclass[a4paper]{article}
\usepackage{ulem}
\usepackage{ctex}
\usepackage{xypic}
\usepackage{amsfonts,amssymb}
\usepackage{multirow}
\usepackage{geometry}
\usepackage{blindtext}
\usepackage{siunitx}
\usepackage{seqsplit}
\usepackage{graphicx}
\usepackage{listings}
\usepackage{lipsum}
\usepackage{courier}
\usepackage{fancyvrb}
\usepackage{etoolbox}

\setlength{\parindent}{0mm}

\linespread{1.0}
\geometry{left=1cm,right=1cm,top=.5cm,bottom=.5cm}

\makeatletter
\patchcmd{\FV@SetupFont}
  {\FV@BaseLineStretch}
  {\fontencoding{T1}\FV@BaseLineStretch}
  {}{}
\makeatother

\renewcommand{\seriesdefault}{\bfdefault}
\lstset{basicstyle=\small\fontencoding{T1}\ttfamily,breaklines=true}
\lstset{numbers=left,frame=shadowbox,tabsize=4}
%\lstset{extendedchars=false}
\begin{document}

\title{The Online Encyclopedia of Integer Sequence (Printed Mirror)}
\author {Africamonkey}
% \maketitle

\twocolumn

\section{简介}

\paragraph{这是一本什么书}

本工具书为 OEIS (http://oeis.org) 的一部分离线镜像,包含了许多常见的整数数列。

\paragraph{本书包含的数列}

本工具书筛选了 OEIS 上符合以下特征的数列:

\begin{itemize}
\item 2017 年 12 月 5 日及之前更新的
\item 编号范围为 A000001 - A296122的
\item 有公式的
\item 单调不下降的
\item 非负的
\end{itemize}

\paragraph{数列的排序方式}

本书数列的排序方式是:设原数列为 $a$ 。将 $a$ 前导 0 的项以及前导 1 的项删去后得到数列 $b$ 。本书将新构造的数列 $b$ 按字典序排序。数列 $b$ 将会在原数列中用下划线标注。

\section{正文}

\footnotesize

\input{../result_sort/tex.txt}

\end{document}
